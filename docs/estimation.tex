% Options for packages loaded elsewhere
\PassOptionsToPackage{unicode}{hyperref}
\PassOptionsToPackage{hyphens}{url}
%
\documentclass[
]{article}
\usepackage{amsmath,amssymb}
\usepackage{lmodern}
\usepackage{iftex}
\ifPDFTeX
  \usepackage[T1]{fontenc}
  \usepackage[utf8]{inputenc}
  \usepackage{textcomp} % provide euro and other symbols
\else % if luatex or xetex
  \usepackage{unicode-math}
  \defaultfontfeatures{Scale=MatchLowercase}
  \defaultfontfeatures[\rmfamily]{Ligatures=TeX,Scale=1}
\fi
% Use upquote if available, for straight quotes in verbatim environments
\IfFileExists{upquote.sty}{\usepackage{upquote}}{}
\IfFileExists{microtype.sty}{% use microtype if available
  \usepackage[]{microtype}
  \UseMicrotypeSet[protrusion]{basicmath} % disable protrusion for tt fonts
}{}
\makeatletter
\@ifundefined{KOMAClassName}{% if non-KOMA class
  \IfFileExists{parskip.sty}{%
    \usepackage{parskip}
  }{% else
    \setlength{\parindent}{0pt}
    \setlength{\parskip}{6pt plus 2pt minus 1pt}}
}{% if KOMA class
  \KOMAoptions{parskip=half}}
\makeatother
\usepackage{xcolor}
\IfFileExists{xurl.sty}{\usepackage{xurl}}{} % add URL line breaks if available
\IfFileExists{bookmark.sty}{\usepackage{bookmark}}{\usepackage{hyperref}}
\hypersetup{
  hidelinks,
  pdfcreator={LaTeX via pandoc}}
\urlstyle{same} % disable monospaced font for URLs
\usepackage[margin=1in]{geometry}
\usepackage{graphicx}
\makeatletter
\def\maxwidth{\ifdim\Gin@nat@width>\linewidth\linewidth\else\Gin@nat@width\fi}
\def\maxheight{\ifdim\Gin@nat@height>\textheight\textheight\else\Gin@nat@height\fi}
\makeatother
% Scale images if necessary, so that they will not overflow the page
% margins by default, and it is still possible to overwrite the defaults
% using explicit options in \includegraphics[width, height, ...]{}
\setkeys{Gin}{width=\maxwidth,height=\maxheight,keepaspectratio}
% Set default figure placement to htbp
\makeatletter
\def\fps@figure{htbp}
\makeatother
\setlength{\emergencystretch}{3em} % prevent overfull lines
\providecommand{\tightlist}{%
  \setlength{\itemsep}{0pt}\setlength{\parskip}{0pt}}
\setcounter{secnumdepth}{-\maxdimen} % remove section numbering
\ifLuaTeX
  \usepackage{selnolig}  % disable illegal ligatures
\fi

\author{}
\date{\vspace{-2.5em}}

\begin{document}

{
\setcounter{tocdepth}{4}
\tableofcontents
}
\hypertarget{maximum-likelihood-estimators}{%
\subsection{Maximum Likelihood
Estimators}\label{maximum-likelihood-estimators}}

After getting a sample from a population with pdf or pmf
\(f(x|\theta)\), it is reasonable to obtain information of \(\theta\).
Thus it is natural to find a good point estimator of \(\theta\) with the
sample, here MLE (Maximum Likelihood Estimators) as a method of point
estimator is useful.

\hypertarget{definition}{%
\subsubsection{Definition}\label{definition}}

Before introducing the definition of MLE, we need to know Point
Estimator and Likelihood Function.

\hypertarget{point-estimator}{%
\paragraph{Point Estimator}\label{point-estimator}}

According to Casella \& Berger(2002), the definition of point estimator
is any function of a sample which means any statistic is a point
estimator.

\hypertarget{likelihood-function}{%
\paragraph{Likelihood Function}\label{likelihood-function}}

Suppose we have random sample \(X_1,X_2,...X_n\) from a population with
pmd or pdf \(p(x|\theta)\), then, given that X=x is observed, the
function of \(\theta\) defined by the joint pdf or pmf
(\(L(\theta|x)=f(x|\theta)\)) is called likelihood function.

Casella \& Berger(2002) said that when considering pdf of pmf
\(f(x|\theta)\), \(\theta\) is fixed and x is the variable. When
considering likelihood function \(L(\theta|x)\), \(\theta\) is variable
and x is the observed sample point. This is the distinction between
joint pdf or pmf and likelihood function.

\hypertarget{definition-of-mle}{%
\paragraph{Definition of MLE}\label{definition-of-mle}}

Intuitively, an event A occurs and others do not show up just because A
has the maximum likelihood. Therefore, the maximum point of likelihood
function should be a good guess of the parameter \(\theta\). This
estimate is called Maximum likelihood estimate.

\hypertarget{process}{%
\subsubsection{Process}\label{process}}

If \(X_1,X_2,...X_n\) are an iid sample from a population with pdf or
pmf p(x\textbar{}\(\theta_1,\theta_2,...\theta_k\)),Likelihood function
is defined by

\[L(\theta|x)=L(\theta_1,\theta_2,...\theta_k|x_1,x_2,...x_n)=\prod_{i=1}^{n}p(x_i|\theta_1,\theta_2,...\theta_k)\]

To find the maximum point of likelihood function, it is equivalent to
find tha maximum point of the log likelihood function
\(l(\theta|x)=logL(\theta|x)\). We just need to solve equation
\(l'(\theta|x)=0\), the solution \(\theta_{mle}\) is the maximum
likelihood estimate of \(\theta\).

Formula of MLE:

\[\theta_{MLE}=\mathop{argmax}\limits _{\theta}\log{L(x|\theta)}\]

\hypertarget{evaluation}{%
\subsubsection{Evaluation}\label{evaluation}}

The advantage of MLE lies in its asymptotic consistency and efficiency.

And there are two drawbacks of MLE. Firstly, how to find and verify that
the global maximum has been found is one drawback (Casella \&
Berger,2002). Though it is an easy problem in calculus, sometimes
difficulties exist due to the densities.

Secondly, we need to take the sensibility of estimate into
consideration. This is a common problem in mathematical maximization
process not just in this case. According to Casella \& Berger(2002), in
MLE, it is unfortune that usually some small changes in sample will lead
to vastly different estimates.

\hypertarget{em-algorithm}{%
\subsection{EM algorithm}\label{em-algorithm}}

\hypertarget{introduction}{%
\subsubsection{Introduction}\label{introduction}}

Given a set of observed data, \(X=[{x_i}]_{i=1}^N\), and model parameter
\(\theta\). Assuming the corresponding unobserved data (latent variable)
\(Z=[{z_i}]_{i=1}^N\), we call \((x_i,z_i)\) as complete data. If we do
not consider the unobserved data, MLE (Maximum Likelihood Estimation)
can be directly used to estimate parameter.

MLE: \[P(X|\theta)\]

\[\theta_{MLE}=\underset {\theta}{argmax} \log P(X|\theta) =\underset {\theta}{argmax} \sum_i^N \log P(x_i|\theta)  \]

Because of unobserved data,

\[P(X|\theta)=\int_Z P(X,Z|\theta)dZ  \]

It is difficult to replace (2) to (1). Thus, EM is introduced. Iteration
is used to get final answer. By EM, expectation step and maximization
step are alternated, until convergence.

\hypertarget{em}{%
\subsubsection{EM}\label{em}}

\[\theta^{t+1}=\underset {\theta}{argmax} \int_Z \log P(X,Z|\theta)\cdot P(Z|X,\theta^t)dZ
=\underset {\theta}{argmax} E_{Z|X,\theta^t} \log P(X,Z|\theta)]\]

\textbf{E-step:}

\[P(Z|X,\theta^t) \rightarrow E_{Z|X,\theta^t} [\log P(X,Z|\theta)]\]

\textbf{M-step:}

\[\theta^{t+1}=\underset {\theta}{argmax} E_{Z|X,\theta^t} [\log P(X,Z|\theta)]\]

\textbf{Prove} procedure (use ELBO and KL Divergence):

\[\log P(X|\theta)=\log P(X,Z|\theta)-\log P(Z|X,\theta) =\log\frac {P(X,Z|\theta)} {q(Z)}-\log\frac{P(Z|X,\theta)} {q(Z)}\]

Integral with \(q(z)\):

\begin{align}

\text{left}&=\int_Z q(Z)\cdot \log P(X|\theta)dZ\\

&=\log P(X|\theta)\int_Z q(Z)dZ\\

&=\log P(X|\theta)\\


\text{right}&=\int_Z q(Z)\cdot log\frac {P(X,Z|\theta)} {q(Z)}dZ-\int_Z q(Z)\cdot log\frac{P(Z|X,\theta)} {q(Z)}dZ\\

\end{align}

where

\[\int_Z q(Z)\cdot \log\frac {P(X,Z|\theta)} {q(Z)}dZ\]

is ELBO (evidence lower bound), and

\[-\int_Z q(Z)\cdot \log\frac{P(Z|X,\theta)} {q(Z)}dZ=\int_Z q(Z)\cdot \log\frac {q(Z)}{P(Z|X,\theta)}dZ\]

is \(KL(q(Z)||P(Z|X,\theta))\).

Correspondingly,

\[KL(q(Z)||P(Z|X,\theta)) \geq 0\]

where 0 is gotten when \(q(Z)=P(Z|X,\theta)\).

Thus,

\[\log P(X|\theta)=ELBO+KL(q|P)\geq ELBO\]

\begin{align}

\hat \theta&=\underset \theta {argmax} ELBO\\

&=\underset \theta {argmax}\int_Z q(Z)\cdot \log\frac {P(X,Z|\theta)} {q(Z)}dZ\\

& =\underset \theta {argmax}\int_Z P(Z|X,\theta)\cdot \log\frac {P(X,Z|\theta)} {P(Z|X,\theta))}dZ \quad \text{because KL=0 is gotten when $q(Z)=P(Z|X,\theta)$)}\\

&= \underset \theta {argmax}\int_Z P(Z|X,\theta)\cdot [\log P(X,Z|\theta)-\log P(Z|X,\theta)]dZ\\

&= \underset \theta {argmax}\int_Z P(Z|X,\theta)\cdot \log P(X,Z|\theta)dZ, \quad \text{because $\log P(Z|X,\theta)$ have nothing to do with $\theta$}

\end{align}

\hypertarget{convergence}{%
\subsubsection{Convergence}\label{convergence}}

We want to find, if \(\theta^t \rightarrow \theta^{t+1}\), when
\[\log P(X|\theta^t)\leq \log P(X|\theta^{t+1})\]

i.e.~will \(\theta\) increase by EM

Prove procedure:

\[\log P(X|\theta)=\log P(X,Z|\theta)-\log P(Z|X,\theta)\] Integral with
z

\begin{align}

\text{left} &=\int_Z P(Z|X,\theta^t)\cdot \log P(X|\theta)dZ\\

& =\log P(X|\theta)\int_Z P(Z|X,\theta^t)dZ\\
 
& = \log P(X|\theta)\\

\text{right}&= \int_Z P(Z|X,\theta^t)\cdot \log P(X,Z|\theta)dZ-\int_Z P(Z|X,\theta^t)\cdot \log P(Z|X,\theta)dZ\\

\end{align}

Denote

\[\int_Z P(Z|X,\theta^t)\cdot \log P(X,Z|\theta)dZ =Q(\theta, \theta^t)\quad \text{and} \quad \int_Z P(Z|X,\theta^t)\cdot \log P(Z|X,\theta)dZ = H(\theta, \theta^t)\]

As

\[\theta^{t+1}=\underset {\theta}{argmax}Q(\theta, \theta^t)\]

Then,

\[Q(\theta^{t+1}, \theta^t)\geq Q(\theta, \theta^t)\]

\(\theta\) is a parameter, let \(\theta=\theta^t\)

\(Q(\theta^{t+1}, \theta^t)\geq Q(\theta^t, \theta^t)\)

Then, we have to prove
\(H(\theta^{t+1}, \theta^t)\leq H(\theta^t, \theta^t)\)

\begin{align}

&H(\theta^{t+1}, \theta^t)-H(\theta^t, \theta^t)\\

=&\int_Z P(Z|X,\theta^t)\cdot \log P(Z|X,\theta^{t+1})dZ-\int_Z P(Z|X,\theta^t)\cdot \log P(Z|X,\theta^t)dZ\\

=&\int_Z P(Z|X,\theta^t)\cdot \log\frac {P(Z|X,\theta^{t+1})}{P(Z|X,\theta^t)}dZ\\
  
\leq & \log \int_Z P(Z|X,\theta^{t+1}) =\log 1= 0 \quad \text{(Because $E[log x]\leq logE[x]$ or use $=-KL(P(Z|X,\theta^t)||P(Z|X,\theta^{t+1})\leq 0)$}

\end{align}

\hypertarget{computerized-adaptive-testingcat}{%
\subsection{Computerized Adaptive
Testing(CAT)}\label{computerized-adaptive-testingcat}}

This section introduces the CAT method.

\hypertarget{fisher-information}{%
\subsubsection{Fisher Information}\label{fisher-information}}

The Fisher Information (MFI) method was introduced by Lord (Lord, 1980;
Thissen \& Mislevy, 2000) and it was the most widespread ISS in the
early days of CAT.

Fisher information is a measurement of the amount of information about
the unknown capacity \(\theta\) generated by the response pattern(Davier
et al., 2019).

\hypertarget{definition-1}{%
\paragraph{Definition}\label{definition-1}}

According to Davier et al.~(2019), firstly, We give the definition of
the first derivative of log likelihood function as Score function:

\[S(X|\theta)=\sum_{i=1}^n\frac{d\log f(X_i;\theta)}{d\theta}\]

where \(f(X_i;\theta)\) refers to the likelihood function, θ is the
underlying latent trait, and x represents the observed response pattern.

Fisher information is second moment of this Score function:

\[I(\theta)=E[S(X|\theta^2)]\] where \(I(\theta)\) is fisher
information.

\hypertarget{mathematical-meanings}{%
\paragraph{Mathematical meanings}\label{mathematical-meanings}}

According to Davier et al.~(2019), it can estimate the variance of the
MLE equation:

As \(E[S(X;\theta)=0\),we can get that

\[I(\theta)=E[S(X|\theta^2)]-E[S(X|\theta)]=Var[S(X|\theta)]\]

It is the expectation of the negative second order derivative of log
likelihood at the true value of the parameter

\[I(\theta)=-E[f''(x|\theta)]=-\int\frac{d^2\log f(x|\theta)f(x|\theta)}{d\theta^2} dx\]

\[I(\theta)=-E\left[\frac  {\partial^2 \log f(x;\theta)}{\partial^2 \theta} \right]\]

Fisher Information reflects the accuracy of our parameter estimates; the
larger it is, the more accurate the parameter estimate, i.e.~the more
information it represents.

\hypertarget{application}{%
\paragraph{Application}\label{application}}

The item k's Fisher information is given by
\(I_k(\theta)=\frac {[P_k'(\theta)]^2}{P_k(\theta)Q_k(\theta)}\)
according to Davier et al.~(2019), where \(P_k(\theta)\) is the item
response function for item k which is specified by the selected IRT
model, and \(Q_k(θ) = 1 − P_k(θ)\), and \(P_k' (θ)\) refers to the first
derivative of the item response function in relation to \(\theta\).

Assuming local independence the test information I(θ) is additive in
item information, that means \(I(\theta)=\Sigma I_k(\theta)\).

For the three-parameter logistic (3PL) model, \(P_j(θ)\) is given by

\[P_k(\theta)=c_k+(1-c_k)\frac{e^{a_k(\theta-b_k)}}{1+e^{a_k(\theta-b_k)}}\]

where \(a_k\), \(b_k\) and \(c_k\) respectively refer to the
discrimination, hardness, and guessing parameter for the kth item.

If the MFI method is applied to item selection, under the current
estimate of \(\theta\) , an eligible item in the bank with the largest
Fisher information will be selected as the next item to be managed.

Since the asymptotic variance of \(\theta ^{ML}\),i.e.~the maximum
likelihood estimate of \(\theta\), is in inverse proportion to the test
information, the MFI method is widely considered to be a method to
minimize the asymptotic variance of the θ estimate, that is, to
asymptotically maximize the measurement precision.

\hypertarget{drawbacks}{%
\paragraph{Drawbacks}\label{drawbacks}}

Firstly, Fisher information does not naturally apply to cognitive
diagnosis as it is by definition on a continuous variable.In the early
phases of CAT, capacity estimation may not yet be accurate. Maximizing
information on the basis of an inaccurate and erratic estimate of
\(\theta\) can be described as ``capitalization on chance''(van der
Linden \& Glas, 2000). Thus, using the MFI in the early stages of a CAT
program may not be ideal.

Secondly, the MFI prefers to pick items with large distinguishing
parameters, but uses few items with smaller discrimination parameters.
This means that some of the items in the item pool may be underutilized.
At the same time,, the excessive exposure of a small number of items
with a high degree of distinction may be a critical threat to the
security of the test(Chang, 2015; Chang \& Ying, 1999).

In addition, the number of items from various content areas or sub-areas
often need to be balanced in order to keep the CAT surface and content
valid (Cheng, Chang, \& Yi, 2007; Yi \& Chang, 2003).

\hypertarget{improvement}{%
\paragraph{Improvement}\label{improvement}}

The global information method was put forward by Chang and Ying (1996),
which use KL distance or information rather than Fisher information in
item selection. They demonstrated that global information is more robust
for addressing the problem of instability in capacity estimation in the
early stage of CAT.

\hypertarget{kl-algorithm}{%
\subsubsection{KL Algorithm}\label{kl-algorithm}}

Chang \& Ying (1996) proposed the global information method which
utilized the KL distance or information instead of Fisher information in
item selection. Being more robust, global information could be used to
combat the instability of ability estimation in the early stage of CAT.

Fisher information is defined on a continuous variable, if involves
discrete, KL Algorithm is preferred.

The Kullback Leibler distance (KL-distance) is defined as a natural
distance function from a ``true'' probability distribution, p, to a
``target'' probability distribution, q. It can be interpreted as the
expected extra message-length per datum due to using a code based on the
wrong (target) distribution compared to using a code based on the true
distribution.

\hypertarget{definition-2}{%
\paragraph{Definition}\label{definition-2}}

For discrete (not necessarily finite) probability distributions, p=\{p1,
\ldots, pn\} and q=\{q1, \ldots, qn\}, the KL-distance is defined to be

\[D_{KL}(P||Q)=\sum_i P(i)\log\left(\frac {P(i)}{Q(i)}\right)\]

For continuous probability densities,

\[D_{KL}(P||Q)=\int_{-\infty}^{\infty} P(x)\log\left(\frac {P(x)}{Q(x)}\right)\]

Xu et al.'s (2005) KL Algorithm:

According to Cover \& Thomas(1991), KL information is a measure of
``distance'' between two probability distributions, which can be defined
as:

\[d[f,g]=E_f\left[\log \frac{f(x)}{g(x)}\right]\] where \(f(x)\) and
\(g(x)\) are two probability distributions.

However, because the unsymmetrical of \(d[f, g]\) and \(d[g, f]\), KL
information is not a real distance measure. KL distance is still
introduced due to the meaning of it, the larger d{[}f, g{]} is
corresponding to the easier it is to single out between the two
probability distributions \(f(x)\) and \(g(x)\) statistically (Henson \&
Douglas, 2005).

\hypertarget{the-kl-algorithm-based-on-kullbackleibler-information-cheng-2009}{%
\paragraph{The KL Algorithm Based on Kullback--Leibler Information
(Cheng,
2009)}\label{the-kl-algorithm-based-on-kullbackleibler-information-cheng-2009}}

Suppose t items are selected, and the available items in the pool form a
set R(t) at this stage. Consider item h in \(R^{(t)}\). In cognitive
diagnosis, conditional distribution of person i's item responses
\(U_{ih}\) given his or her latent state, or cognitive profile,
\(\alpha_i\) are what interested. According to the notation of McGlohen
and Chang (2008),
\(\alpha_{i}=(\alpha_{i1},\alpha_{i2},...,\alpha_{ik},...,\alpha_{iK})'\).

Here \(\alpha_{ik}\) = 0 indicates that the \(i\)th examinee not masters
the kth attribute and \(\alpha_{ik} = 1\) otherwise. An attribute is a
task, cognitive process, or skill involved in answering an item.

Due to the unknown true state, a global measure of discrimination can be
constructed on the basis of the KL distance between the distribution of
\(U_{ih}\) given the current estimate of person \(i\)'s latent cognitive
state (i.e., \(f(U_{ih}|\hat \alpha_i^{(t)}\))) and the distribution of
\(U_{ih}\) given other states.

The KL distance between \(f(U_{ih}|\hat \alpha_i^{(t)}\))) and the
conditional distribution of \(U_{ih}\) given another latent state
\(\alpha_c\), i.e., \(f(U_{ih}|\alpha_c\))), can be computed as follows:

\[D_h(\hat \alpha_i^{(t)}||\alpha_c)=\sum_{q=0}^1\log \left[\frac{P(U_{ih}=q|\hat \alpha_i^{(t)})}{P(U_{ih}=q|\alpha_c)}\right]P(U_{ih}=q|\hat \alpha_i^{(t)})\]

Xu et al.~(2003) stated using the straight sum of the KL distances
between \(f(U_{ih}|\hat \alpha_i^{(t)}\))) and all the
\(f(U_{ih}|\alpha_c\))), c = 1, 2,\ldots, \(2^K\) (when there are K
attributes, there are \(2^K\) possible latent cognitive states):

\[KL_h(\hat \alpha_i^{(t)})=\sum _{c=1}^{2^K}D_h(\hat \alpha_i^{(t)}||\alpha_c)\]

Then the \((t + 1)\)th item for the \(i\)th examinee is the item in
\(R(t)\) that maximizes \(KL_h(\hat \alpha_i^{(t)})\). This is referred
to as the KL algorithm. The items selected using this algorithm are the
most powerful ones on average in distinguishing the current latent class
estimate from all other possible latent classes.

\hypertarget{use-of-kl-distance}{%
\paragraph{Use of KL Distance}\label{use-of-kl-distance}}

It is helpful to choose the optimal parameter. For instance, if p(x) is
unknown, a \(q(x|\theta)\) can be constructed to estimate p(x). In order
to know \(\theta\), select N samples from p(x) and construct such
function:

\[D_{KL}(p||q)=\sum_{i=1}^Np(x_i)(\log p(x_i)-\log(q(x_i|\theta))\]

Then use MLE to estimate \(\theta\).

\hypertarget{shannon-entropy}{%
\subsubsection{Shannon entropy}\label{shannon-entropy}}

It is necessary to know the uncertainty of a random variable and Shannon
entropy is a good candidate to measure the uncertainty. Cheng(2009)
listed an example about the Shannon entropy: a fair coin has entropy of
one unit while an unfair coin has lower entropy because there is less
uncertainty when guessing the outcome of one unfair coin.

\hypertarget{definition-3}{%
\paragraph{Definition}\label{definition-3}}

For a discrete random variable X which takes value among
\(x_1,x_2,...x_n\), the Shannon entropy is defined as:

\[H(X)=-\sum_{i=1}^np(x_i)\log_b(x_i)\]

In the definition, \(p(x_i)\) is the probability when X = \(x_i\).
\(H(X)\) can also be written as \(H(P)\) or H(\(p_1,p_2,...p_n\)). Owing
to the formula, we can conclude that independent uncertainties are
additive. \(b\) is the base of logarithm, which takes value among 2,e
and 10. The differences among choices of b are the unit of entropy. For
\(b=2\), unit is bit; for \(b = e\), unit is nat; for \$ b = 10\$, unit
is dit or digit.

\hypertarget{properties}{%
\subsubsection{Properties}\label{properties}}

The choice of b does not influence the properties of Shannon entropy, so
we do not need care the value of b in this part.

\hypertarget{nonnegativity}{%
\paragraph{Nonnegativity}\label{nonnegativity}}

\[H(p_1,p_2,...p_n)\ge0\]

This equality holds only when the distribution is certain which means
\(\exists\quad p_i=1\), and other \(p_j=0\), where \(i\neq j\).

\hypertarget{maximality}{%
\paragraph{Maximality}\label{maximality}}

\(H(P)\) reaches its maximum when all the events share same probability,
which means \(p_1=p_2=p_n=\frac{1}{n}\). Moreover, if the events all
share same probability, \(H(P)\) increases with the number of events
increasing. It means

\[H(\frac{1}{n},\frac{1}{n},...)<H(\frac{1}{n+1},\frac{1}{n+1}...)\]

\hypertarget{concavity}{%
\paragraph{Concavity}\label{concavity}}

Shannon entropy is a concave function of P.

\hypertarget{continuity}{%
\paragraph{Continuity}\label{continuity}}

Shannon entropy is a continuous function of P which means small changes
on probability lead to small changes on entropy.

\hypertarget{symmetry}{%
\paragraph{Symmetry}\label{symmetry}}

\[H(P_1)=H(P_2)\] if \(P_1,P_2\) are different permutations of one
probability distribution.

\hypertarget{expansible}{%
\paragraph{Expansible}\label{expansible}}

\[H(p_1,p_2,...p_n)=H(p_1,p_2,...p_n,0)\]

\hypertarget{difference-between-kl-divergence-and-shannon-entropy}{%
\subsubsection{Difference between KL-Divergence and Shannon
Entropy}\label{difference-between-kl-divergence-and-shannon-entropy}}

Shannon Entropy is used to measure the uncertainty of one probability
distribution, while KL-Distance is used to measure the divergence
between two probability distributions.So, in the use of Shannon Entropy,
only one probability distribution will be involved with two involved in
KL-Divergence.

\hypertarget{reference}{%
\subsection{Reference}\label{reference}}

\begin{itemize}
\item
  Casella,G.,\& Berger R.L.,(2002). Statistical Inference.USA.RR
  Donnelley
\item
  Chang, H. H. (2015). \{Psychometrics behind computerized adaptive
  testing\}. Psychometrika, 80(1),1--20.
\item
  Chang, H. H., \& Ying, Z. (1999). \{a-Stratified multistage
  computerized adaptive testing. Applied Psychological Measurement\},
  23(3), 211--222.
\item
  Chang, H.-H., \& Ying, Z.(1996) `A Global Information Approach to
  Computerized Adaptive Testing',
  \textit{Applied Psychological Measurement}, 20(3), pp.~213-229.
  \url{doi:10.1177/014662169602000303}
\item
  Cheng, Y. (2009). Computerized adaptive testing for cognitive
  diagnosis. In D. J. Weiss (Ed.), Proceedings of the 2009 GMAC
  Conference on Computerized Adaptive Testing. Retrieved {[}5 Aug
  2022{]} from www.psych.umn.edu/psylabs/CATCentral/
\item
  Cheng, Y (2009) `When Cognitive Diagnose Meets Computerized Adaptive
  Testing:CD-CAT', \emph{Psychometrika},VOL. 74, NO. 4, 619--632
\item
  Cheng, Y., Chang, H. H., \& Yi, Q. (2007). \{Two-phase item selection
  procedure for flexible content balancing in CAT\}. Applied
  Psychological Measurement, 31(6), 467--482.
\item
  Davier, M.V. et al.~(2019). \{Methodology of Educational Measurement
  and Assessment\}. Available at:
  \url{https://doi.org/10.1007/978-3-030-05584-4} (Downloaded: 18 July
  2022)
\item
  Henson, R. \& Douglas J. (2005). Test construction for cognitive
  diagnosis. \emph{Applied Psychological Measurement}, 29, 262-277.
\item
  Lord, F. M. (1980).
  \textit{Applications of item response theory to practical testing problems}.
  Hillsdale,NJ: Erlbaum.
\item
  McGlohen, M.K., \& Chang, H. (2008). Combining computer adaptive
  testing technology with cognitively diagnostic assessment.
  \emph{Behavioral Research Methods}, 40, 808--821.
\item
  Thissen, D., \& Mislevy, R. J. (2000). \{Testing algorithm. In H.
  Wainer \& N. J. Dorans (Eds.), Computerized adaptive testing: A
  primer\}. Hillsdale, NJ: Erlbarm.
\item
  Van der Linden, W. J., \& Glas, C. A. W. (2000). \{Capitalization on
  item calibration error in adaptive testing\}. Applied Measurement in
  Education, 13(1), 35--53.
\item
  Xu, X., Chang, H., \& Douglas, J. (2005). Computerized adaptive
  testing strategies for cognitive diagnosis. Paper presented at the
  annual meeting of National Council on Measurement in Education,
  Montreal, Canada.
\item
  Yi, Q., \& Chang, H. H. (2003). \{a-Stratified CAT design with content
  blocking\}. British Journal of Mathematical and Statistical
  Psychology, 56(2), 359--378.
\end{itemize}

\end{document}
